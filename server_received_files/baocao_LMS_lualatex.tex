
\documentclass[a4paper,12pt]{article}
\usepackage{fontspec}
\setmainfont{Times New Roman}
\usepackage{geometry}
\geometry{margin=2.5cm}
\usepackage{titlesec}
\usepackage{setspace}
\usepackage{hyperref}
\usepackage{listings}
\usepackage{xcolor}

\lstset{
  basicstyle=\ttfamily\footnotesize,
  backgroundcolor=\color{gray!10},
  frame=single,
  breaklines=true
}

\setstretch{1.3}

\title{\textbf{Xây dựng nền tảng số hỗ trợ học tập cho sinh viên dựa trên LMS và phân tích dữ liệu}}
\author{Nguyễn Nam}
\date{2025}

\begin{document}
\maketitle

\begin{abstract}
Trong bối cảnh chuyển đổi số giáo dục đang diễn ra mạnh mẽ, việc xây dựng các nền tảng học tập trực tuyến (Learning Management System – LMS) đã trở thành một trong những hướng tiếp cận chủ đạo nhằm nâng cao hiệu quả giảng dạy và học tập. Đề tài này tập trung vào việc phát triển một hệ thống quản lý học tập trực tuyến có khả năng theo dõi, phân tích và gợi ý học tập cho sinh viên dựa trên dữ liệu hành vi học tập.

Hệ thống được xây dựng bằng Flask (Python) với cơ sở dữ liệu SQLite, tích hợp các công nghệ Learning Analytics và AI rule-based để đưa ra các gợi ý học tập phù hợp cho từng người dùng. Ứng dụng cung cấp các chức năng chính như: đăng ký/đăng nhập, quản lý khóa học, nộp bài tập, chấm điểm, thống kê học tập và tư vấn học tập cá nhân hóa. Giao diện được thiết kế thân thiện với người dùng bằng Bootstrap 5 và trực quan hóa dữ liệu bằng Chart.js.

Kết quả đạt được cho thấy hệ thống hoạt động ổn định, đáp ứng được nhu cầu cơ bản của một LMS hiện đại, đồng thời thể hiện khả năng mở rộng trong việc ứng dụng trí tuệ nhân tạo và phân tích dữ liệu vào giáo dục đại học.
\end{abstract}

\section{Giới thiệu}
Giáo dục đại học hiện nay đang dần chuyển đổi từ mô hình truyền thống sang mô hình học tập kết hợp (blended learning) và học tập trực tuyến (e-learning). Việc áp dụng công nghệ thông tin vào giảng dạy không chỉ giúp tối ưu hóa thời gian và nguồn lực mà còn cho phép phân tích hành vi học tập của sinh viên, từ đó đưa ra các gợi ý học tập cá nhân hóa.

Các hệ thống LMS như Moodle, Canvas hay Google Classroom đã được triển khai rộng rãi, tuy nhiên các hệ thống này thường phức tạp, khó tùy chỉnh và yêu cầu hạ tầng mạnh. Với mục tiêu tạo ra một mô hình nhẹ, dễ triển khai, phù hợp với môi trường học tập trong nước, đề tài lựa chọn xây dựng hệ thống LMS đơn giản nhưng đầy đủ chức năng cần thiết, kết hợp thêm thành phần AI và phân tích dữ liệu học tập để hỗ trợ sinh viên và giảng viên trong quá trình học tập.

Nền tảng được kỳ vọng không chỉ phục vụ mục tiêu học tập mà còn là bước đệm cho các nghiên cứu chuyên sâu hơn trong lĩnh vực Learning Analytics và AI trong giáo dục (AI in Education).

\section{Nghiên cứu liên quan}
Trong những năm gần đây, nhiều nghiên cứu về Learning Management System (LMS) và Learning Analytics (LA) đã được triển khai nhằm hỗ trợ việc ra quyết định trong giáo dục. Các hệ thống LMS phổ biến như Moodle, Blackboard hay Canvas cung cấp đầy đủ chức năng quản lý khóa học, bài tập, điểm số và đánh giá sinh viên. Tuy nhiên, chúng thường thiếu tính linh hoạt trong việc tùy chỉnh hoặc mở rộng cho các mô hình AI thử nghiệm.

Hướng tiếp cận Learning Analytics được giới thiệu bởi Siemens (2013) và Ferguson (2012) nhấn mạnh tầm quan trọng của việc thu thập, phân tích và trực quan hóa dữ liệu học tập để hiểu rõ hơn về quá trình học của người học. Từ đó, các mô hình early warning system hay personalized recommendation được áp dụng nhằm dự đoán sinh viên có nguy cơ học yếu hoặc đưa ra chiến lược học tập phù hợp.

Đề tài kế thừa các hướng nghiên cứu này, đồng thời xây dựng một hệ thống thử nghiệm phù hợp với quy mô nhỏ, sử dụng phương pháp rule-based AI để đưa ra gợi ý học tập mà không cần đến mô hình học máy phức tạp, giúp dễ dàng triển khai trong môi trường học tập tại trường đại học.

\section{Tổng quan nghiên cứu của đề tài}
Hệ thống “EduLearn” – nền tảng số hỗ trợ học tập – được thiết kế theo pipeline gồm các bước:
\begin{enumerate}
    \item \textbf{Xây dựng cơ sở dữ liệu (Database Design):} gồm 6 bảng chính: users, courses, enrollments, assignments, submissions, logs.
    \item \textbf{Phát triển backend:} sử dụng Flask, Flask-Login, SQLAlchemy.
    \item \textbf{Phát triển frontend:} sử dụng Bootstrap 5 và Chart.js.
    \item \textbf{Phân tích dữ liệu học tập:} áp dụng pandas để tính toán các chỉ số.
    \item \textbf{Tích hợp AI tư vấn học tập:} sử dụng rule-based AI để gợi ý cho sinh viên.
    \item \textbf{Đánh giá và thử nghiệm:} thử nghiệm với dữ liệu mẫu gồm 80 sinh viên và 6 giảng viên.
\end{enumerate}

\section{Phương pháp và kỹ thuật sử dụng}
\begin{itemize}
    \item \textbf{Ngôn ngữ và Framework:} Python 3.x, Flask
    \item \textbf{Cơ sở dữ liệu:} SQLite
    \item \textbf{Công nghệ giao diện:} HTML5, CSS3, Bootstrap 5, Chart.js
    \item \textbf{Phân tích dữ liệu:} pandas, matplotlib
    \item \textbf{Xác thực và phân quyền:} Flask-Login
    \item \textbf{AI hỗ trợ học tập:} rule-based system
\end{itemize}

\noindent\textbf{Ví dụ mã:}
\begin{lstlisting}[language=Python]
def advice(score, login_count):
    if score < 5:
        return "Bạn nên ôn lại các chương cơ bản và làm thêm bài tập."
    elif login_count < 3:
        return "Bạn cần dành thêm thời gian học, đăng nhập thường xuyên hơn."
    else:
        return "Bạn đang học khá tốt, tiếp tục phát huy!"
\end{lstlisting}

\section{Dữ liệu và nguồn dữ liệu}
Dữ liệu được khởi tạo tự động trong cơ sở dữ liệu SQLite thông qua hàm \texttt{create\_sample\_data()}. Bao gồm:
\begin{itemize}
    \item 6 giảng viên và 80 sinh viên
    \item 6 khóa học như Nhập môn Python, Machine Learning cơ bản, Phân tích dữ liệu với Pandas
    \item Logs hoạt động: đăng nhập, xem tài liệu, nộp bài
\end{itemize}

\section{Triển khai và kết quả}
Hệ thống được triển khai dưới dạng ứng dụng Flask có thể chạy cục bộ bằng lệnh:
\begin{lstlisting}[language=bash]
python app.py
\end{lstlisting}

Sau khi khởi động, hệ thống hiển thị các chỉ số tổng hợp, tỷ lệ hoàn thành bài tập, phân bố điểm và biểu đồ hoạt động học tập. Hệ thống AI rule-based gợi ý cho sinh viên và hỗ trợ giảng viên theo dõi tiến độ.

\section{Kết luận và hướng phát triển}
Hệ thống LMS được xây dựng đáp ứng các mục tiêu: quản lý học tập, hỗ trợ phân tích dữ liệu và tư vấn học tập tự động. Trong tương lai, hệ thống có thể mở rộng với:
\begin{itemize}
    \item Tích hợp Machine Learning để dự đoán nguy cơ rớt học
    \item Bổ sung tính năng chat realtime và ứng dụng di động
    \item Kết hợp AI ngôn ngữ tự nhiên (LLM) để tạo phản hồi linh hoạt hơn
\end{itemize}

\end{document}
